% page layout
\geometry{
  dvips=false, pdftex=false, vtex=false, % drivers can have unexpected behaviors
  papersize={8in,10in}, % specified by MIT Press
  centering, % split margins equally
  margin=.75in, % margins (must be at least .5in on all sides)
  includemp, includehead % include sidenotes and header within body
  % showframe % reveal page structure (only for testing/debugging)
}
% put on letter page, show frame around print area
\usepackage[frame,letter,center,noinfo]{crop}

% headers/footers
\renewcommand{\chaptermark}[1]{\markboth{#1}{}} % prevent chaptermark from uppercasing
% left header on even pages
\fancyhead[LE]{\sffamily\thepage\quad\textsc{\plaintitle}} % page number then book title
% right header on odd pages
\fancyhead[RO]{%
  \sffamily
  \ifnum\value{chapter}>0 \thechapter\hspace{3pt} \fi % include chapter number if not 0
  \textsc{\leftmark}\quad\thepage % chapter title then page number
}

% part and chapter headings
\titleformat{\part}%
  [hang]% shape
  {\sffamily\scshape\Huge\center}% format applied to label+text
  {\thepart}% label
  {.5em}% horizontal separation between label and title body
  {} % before the title body
  
\titleformat{\chapter}%
  [hang]% shape
  {\sffamily\scshape\huge}% format applied to label+text
  {\thechapter}% label
  {.5em}% horizontal separation between label and title body
  {} % before the title body

% table of contents
\usepackage[titles]{tocloft}
% part fonts
\renewcommand{\cftpartfont}{\sffamily\scshape\Large}
\renewcommand{\cftpartpagefont}{\sffamily\scshape\Large}
% % part layout
% chapter fonts
\renewcommand{\cftchapfont}{\sffamily\scshape\Large}
\renewcommand{\cftchappagefont}{\sffamily\scshape\Large}
% chapter layout
\renewcommand{\cftchapnumwidth}{1em}
\setlength{\cftchapindent}{2em}
\setlength{\cftbeforechapskip}{.5em}

% lists
\usepackage{enumitem}
\setlist[itemize]{
  label={--}, % en-dash as bullet symbol
  topsep=0em % no space before/after list
}

% colors (solarized) (must be in cmyk)
\definecolor{.yellow}{cmyk}{0,.24,1,.29}
\definecolor{.orange}{cmyk}{0,.63,.89,.20}
\definecolor{.red}{cmyk}{0,.77,.79,.14}
\definecolor{.magenta}{cmyk}{0,.74,.38,.17}
\definecolor{.violet}{cmyk}{.45,.42,0,.23}
\definecolor{.blue}{cmyk}{.82,.34,0,.18}
\definecolor{.cyan}{cmyk}{7.4,0,.06,.37}
\definecolor{.green}{cmyk}{.13,0,1,.40}
\definecolor{.base0}{cmyk}{.13,.01,0,.41}

% icons (not using fontawesome5 package because some needed icons are from 6)
\usepackage{fontspec}
\newfontfamily{\fa}[Path=fonts/]{Font Awesome 6 Free-Solid-900.otf}
\def\faAppleWhole{{\fa \symbol{"F5D1}}}
\def\faMicroscope{{\fa \symbol{"F610}}}
\def\faLeaf{{\fa \symbol{"F06C}}}
\def\faPersonFallingBurst{{\fa \symbol{"E547}}}
\def\faCode{{\fa \symbol{"F121}}}
\def\faMagnifyingGlassPlus{{\fa \symbol{"F00E}}}
\def\faPenRuler{{\fa \symbol{"F5AE}}}
\def\faComments{{\fa \symbol{"F086}}}
\def\faBook{{\fa \symbol{"F02D}}}

% boxes
\usepackage{tcolorbox}
\tcbuselibrary{skins} % allow changing skin
\tcbuselibrary{breakable} % allow breaking across pages

% set options for all boxes
\tcbset{enhanced jigsaw, % skin
        breakable, % break across pages
        boxrule=2pt, % border width
        colback=white, % background color
        fonttitle=\sffamily\scshape\LARGE, % title sans serif and small caps
        left=2mm, right=2mm, bottom=1mm, top=1mm, % padding inside box
        toptitle=1mm, bottomtitle=1mm, % spacing above/below title
        before upper={\justifying\parindent1.0pc}} % font changes inside

% define box for each type (color, icon, title)
\newtcolorbox{learning_goals}{%
  colframe=.red, title=\faAppleWhole \enspace Learning goals}
\newtcolorbox{case_study}{%
  colframe=.blue, title=\faMicroscope \enspace Case study}
\newtcolorbox{ethical_considerations}{%
  colframe=.green, title=\faLeaf \enspace Ethical considerations}
\newtcolorbox{accident_report}{%
  colframe=.orange, title=\faPersonFallingBurst \enspace Accident report}
\newtcolorbox{code}{%
  colframe=.base0, title=\faCode \enspace Code}
\newtcolorbox{depth}{%
  colframe=.violet, title=\faMagnifyingGlassPlus \enspace Depth}
\newtcolorbox{exercises}{%
  colframe=.yellow, title=\faPenRuler \enspace Exercises}
\newtcolorbox{discussion_questions}{%
  colframe=.cyan, title=\faComments \enspace Discussion questions}
\newtcolorbox{readings}{%
  colframe=.magenta, title=\faBook \enspace Readings}

% define within-box title as subsection
\newcommand{\boxtitle}[1]{\subsection*{#1}}

% define box for figures within boxes
\usepackage{caption}
\newtcolorbox[blend into=figures]{boxfigure}[1][]{%
  boxrule=0pt,colframe=white,colback=white,#1} %title={#2},

% for testing/debugging
\usepackage{lipsum}
